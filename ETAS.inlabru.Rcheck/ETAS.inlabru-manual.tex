\nonstopmode{}
\documentclass[letterpaper]{book}
\usepackage[times,inconsolata,hyper]{Rd}
\usepackage{makeidx}
\usepackage[utf8]{inputenc} % @SET ENCODING@
% \usepackage{graphicx} % @USE GRAPHICX@
\makeindex{}
\begin{document}
\chapter*{}
\begin{center}
{\textbf{\huge Package `ETAS.inlabru'}}
\par\bigskip{\large \today}
\end{center}
\inputencoding{utf8}
\ifthenelse{\boolean{Rd@use@hyper}}{\hypersetup{pdftitle = {ETAS.inlabru: This package uses inlabru to implement a Bayesian ETAS model for modelling seismic sequences}}}{}
\begin{description}
\raggedright{}
\item[Type]\AsIs{Package}
\item[Title]\AsIs{This package uses inlabru to implement a Bayesian ETAS model for
modelling seismic sequences}
\item[Version]\AsIs{0.1.0}
\item[Author]\AsIs{Mark Naylor, Francesco Serafini, and Finn Lindgren}
\item[Maintainer]\AsIs{The package maintainer }\email{mark.naylor@ed.ac.ukt}\AsIs{}
\item[Description]\AsIs{Modelling and inversion of ETAS model of seismicity using inlabru
The Epidemic Type Aftershock Sequence (ETAS) model is designed to 
model earthquakes that are triggered by previous events. In statistics, this is
referred to as a Hawkes process.
The code can be used to generate synthetic ETAS catalogues which can also
include some seeded events to model specific sequences.
We also implement a Bayesian inversion scheme using the Integrated Nested Laplace
Approximation (INLA) using inlabru. 
For the temporal model, given a training catalogue of times and magnitudes, the code 
returns the joint posteriors for all the ETAS parameters.
In the future roadmap, we will include tools to model the spatial distribution
and spatio-temporal evolution of seismic sequences.}
\item[License]\AsIs{What license is it under?}
\item[Encoding]\AsIs{UTF-8}
\item[LazyData]\AsIs{true}
\item[RoxygenNote]\AsIs{7.2.2}
\item[Roxygen]\AsIs{list(markdown = TRUE)}
\item[Imports]\AsIs{lemon, parallel, tidyquant, dplyr, ggplot2, foreach, INLA,
inlabru}
\item[NeedsCompilation]\AsIs{no}
\end{description}
\Rdcontents{\R{} topics documented:}
\inputencoding{utf8}
\HeaderA{breaks\_exp}{Title}{breaks.Rul.exp}
%
\begin{Description}\relax
Title
\end{Description}
%
\begin{Usage}
\begin{verbatim}
breaks_exp(tt_, T2_, coef_ = 2, delta_, N_exp_ = 10)
\end{verbatim}
\end{Usage}
%
\begin{Arguments}
\begin{ldescription}
\item[\code{tt\_}] List of the event times a grid is needed for \LinkA{days}{days}.

\item[\code{T2\_}] End of temporal domain \LinkA{days}{days}.

\item[\code{coef\_}] TimeBinning parameter:

\item[\code{delta\_}] TimeBinning parameter:

\item[\code{N\_exp\_}] TimeBinning parameter: Number of bins
\end{ldescription}
\end{Arguments}
\inputencoding{utf8}
\HeaderA{compute.grid}{Title}{compute.grid}
%
\begin{Description}\relax
Title
\end{Description}
%
\begin{Usage}
\begin{verbatim}
compute.grid(param., list.input_)
\end{verbatim}
\end{Usage}
%
\begin{Arguments}
\begin{ldescription}
\item[\code{list.input\_}] 
\end{ldescription}
\end{Arguments}
\inputencoding{utf8}
\HeaderA{create.input.list.temporal.noCatalogue}{Function to create a default input file for the ETAS Hawkes temporal model where no catalogue is specified in the input file}{create.input.list.temporal.noCatalogue}
%
\begin{Description}\relax
Function to create a default input file for the ETAS Hawkes temporal model where no catalogue is specified in the input file
\end{Description}
%
\begin{Usage}
\begin{verbatim}
create.input.list.temporal.noCatalogue(input_path)
\end{verbatim}
\end{Usage}
%
\begin{Arguments}
\begin{ldescription}
\item[\code{input\_path}] Input file and path as a string
\end{ldescription}
\end{Arguments}
%
\begin{Value}
The formatted input.list with the elements required for the temporal Hawkes model
\end{Value}
%
\begin{Examples}
\begin{ExampleCode}
# HOW DO WE REFERENCE A FILE IN THE data DIRECTORY?
#create.input.list.temporal.noCatalogue('data/user_input_synthetic_noCatalog.txt')
\end{ExampleCode}
\end{Examples}
\inputencoding{utf8}
\HeaderA{create.input.list.temporal.withCatalogue}{Function to create a default input file for the ETAS Hawkes temporal model where a catalogue is specified in the input file.}{create.input.list.temporal.withCatalogue}
%
\begin{Description}\relax
Function to create a default input file for the ETAS Hawkes temporal model where a catalogue is specified in the input file.
\end{Description}
%
\begin{Usage}
\begin{verbatim}
create.input.list.temporal.withCatalogue(input_path)
\end{verbatim}
\end{Usage}
%
\begin{Arguments}
\begin{ldescription}
\item[\code{input\_path}] 
\end{ldescription}
\end{Arguments}
\inputencoding{utf8}
\HeaderA{gamma.t}{Gamma copula transformation: Conversion of ETAS para to internal scale}{gamma.t}
%
\begin{Description}\relax
Gamma copula transformation: Conversion of ETAS para to internal scale
\end{Description}
%
\begin{Usage}
\begin{verbatim}
## S3 method for class 't'
gamma(x, a, b)
\end{verbatim}
\end{Usage}
%
\begin{Arguments}
\begin{ldescription}
\item[\code{b}] 
\end{ldescription}
\end{Arguments}
\inputencoding{utf8}
\HeaderA{generate.temporal.ETAS.synthetic}{Generates a sythetic catalogue using the ETAS model}{generate.temporal.ETAS.synthetic}
%
\begin{Description}\relax
Generates a sythetic catalogue using the ETAS model
\end{Description}
%
\begin{Usage}
\begin{verbatim}
generate.temporal.ETAS.synthetic(
  theta,
  beta.p,
  M0,
  T1,
  T2,
  Ht = NULL,
  ncore = 1
)
\end{verbatim}
\end{Usage}
%
\begin{Arguments}
\begin{ldescription}
\item[\code{theta}] ETAS parameters \code{data.frame(mu=mu, K=K, alpha=alpha, c=c, p=p)}.

\item[\code{beta.p}] Slope of GR relation: beta = b ln(10).

\item[\code{M0}] The minimum magnitude in the synthetic catalogue.

\item[\code{T1}] The start time for the synthetic catalogue \LinkA{days}{days}.

\item[\code{T2}] The end time for the synthetic catalogue \LinkA{days}{days}.

\item[\code{Ht}] A catalogue history to impose on the synthetic sequence.

\item[\code{ncore}] Integer number of compute cores to use.
\end{ldescription}
\end{Arguments}
%
\begin{Value}
A data.frame of the temporal catalogue with columns \AsIs{\texttt{[t\_i, M\_i, gen\_i]}}
where, \code{t\_i} are the times,  \code{M\_i} the magnitudes, \code{gen\_i} includes information about the generation number
\end{Value}
%
\begin{Examples}
\begin{ExampleCode}
## EXAMPLE 1: Generate a 1000 day synthetic ETAS catalogue

generate.temporal.ETAS.synthetic( theta=data.frame(mu=0.1, K=0.089, alpha=2.29, c=0.11, p=1.08), beta.p=log(10), M0=2.5, T1=0, T2=1000 )


## EXAMPLE 2: To generate a 1000 day catalogue including a M6.7 event on day 500

Ht <- data.frame(ts=c(500), magnitudes=c(6.7))
generate.temporal.ETAS.synthetic( theta=data.frame(mu=0.1, K=0.089, alpha=2.29, c=0.11, p=1.08), beta.p=log(10), M0=2.5, T1=0, T2=1000, Ht=Ht )
\end{ExampleCode}
\end{Examples}
\inputencoding{utf8}
\HeaderA{get\_posterior\_N}{Title}{get.Rul.posterior.Rul.N}
%
\begin{Description}\relax
Title
\end{Description}
%
\begin{Usage}
\begin{verbatim}
get_posterior_N(input.list)
\end{verbatim}
\end{Usage}
%
\begin{Arguments}
\begin{ldescription}
\item[\code{input.list}] 
\end{ldescription}
\end{Arguments}
\inputencoding{utf8}
\HeaderA{get\_posterior\_param}{Generate summary information on the fitted ETAS model}{get.Rul.posterior.Rul.param}
%
\begin{Description}\relax
Generate summary information on the fitted ETAS model
\end{Description}
%
\begin{Usage}
\begin{verbatim}
get_posterior_param(input.list)
\end{verbatim}
\end{Usage}
%
\begin{Arguments}
\begin{ldescription}
\item[\code{input.list}] Which has combined the input file (for link functions) and bru output (for marginals)
\end{ldescription}
\end{Arguments}
%
\begin{Value}
Data frame summary and summary plot
\end{Value}
\inputencoding{utf8}
\HeaderA{gt}{Time triggering function - used by bayesianETAS Used for comparing output of inlabru with Bayesian ETAS MN: TODO Cross-reference the paper}{gt}
%
\begin{Description}\relax
Time triggering function - used by bayesianETAS
Used for comparing output of inlabru with Bayesian ETAS
MN: TODO Cross-reference the paper
\end{Description}
%
\begin{Usage}
\begin{verbatim}
gt(th, t, ti, mi, M0)
\end{verbatim}
\end{Usage}
%
\begin{Arguments}
\begin{ldescription}
\item[\code{M0}] Minimum magnitude threshold
\end{ldescription}
\end{Arguments}
\inputencoding{utf8}
\HeaderA{gt.2}{Time triggering function - used by Inlabru MN: TODO Cross-reference the paper}{gt.2}
%
\begin{Description}\relax
Time triggering function - used by Inlabru
MN: TODO Cross-reference the paper
\end{Description}
%
\begin{Usage}
\begin{verbatim}
gt.2(th, t, ti, mi, M0)
\end{verbatim}
\end{Usage}
%
\begin{Arguments}
\begin{ldescription}
\item[\code{M0}] Minimum magnitude threshold
\end{ldescription}
\end{Arguments}
\inputencoding{utf8}
\HeaderA{Int.ETAS.time.trig.function}{Integrated ETAS time-triggering function}{Int.ETAS.time.trig.function}
%
\begin{Description}\relax
Integrated ETAS time-triggering function
\end{Description}
%
\begin{Usage}
\begin{verbatim}
Int.ETAS.time.trig.function(theta, th, T2)
\end{verbatim}
\end{Usage}
%
\begin{Arguments}
\begin{ldescription}
\item[\code{theta}] ETAS parameters data.frame(mu=mu, K=K, alpha=alpha, c=c, p=p)

\item[\code{th}] Time of past event? \LinkA{days}{days}

\item[\code{T2}] End of temporal model domain.
\end{ldescription}
\end{Arguments}
\inputencoding{utf8}
\HeaderA{IntInjecIntensity}{Title}{IntInjecIntensity}
%
\begin{Description}\relax
Title
\end{Description}
%
\begin{Usage}
\begin{verbatim}
IntInjecIntensity(a = 50, V.i = 1, tau = 10, T.i, T2)
\end{verbatim}
\end{Usage}
%
\begin{Arguments}
\begin{ldescription}
\item[\code{a}] Event rate per unit volume injected

\item[\code{V.i}] Injected volume

\item[\code{tau}] Decau rate \LinkA{days}{days}

\item[\code{T.i}] Time of injection event

\item[\code{T2}] 
\end{ldescription}
\end{Arguments}
\inputencoding{utf8}
\HeaderA{inv.exp.t}{Inverse exponential link function:}{inv.exp.t}
%
\begin{Description}\relax
Inverse exponential link function:
\end{Description}
%
\begin{Usage}
\begin{verbatim}
inv.exp.t(x, rate)
\end{verbatim}
\end{Usage}
%
\begin{Arguments}
\begin{ldescription}
\item[\code{rate}] 
\end{ldescription}
\end{Arguments}
\inputencoding{utf8}
\HeaderA{inv.gamma.t}{Inverse gamma copula transformation:}{inv.gamma.t}
%
\begin{Description}\relax
Inverse gamma copula transformation:
\end{Description}
%
\begin{Usage}
\begin{verbatim}
inv.gamma.t(x, a, b)
\end{verbatim}
\end{Usage}
%
\begin{Arguments}
\begin{ldescription}
\item[\code{b}] 
\end{ldescription}
\end{Arguments}
\inputencoding{utf8}
\HeaderA{Inv.Int.ETAS.time.trig.function}{Inverse of integrated ETAS time-triggering function}{Inv.Int.ETAS.time.trig.function}
%
\begin{Description}\relax
Inverse of integrated ETAS time-triggering function
\end{Description}
%
\begin{Usage}
\begin{verbatim}
Inv.Int.ETAS.time.trig.function(theta, omega, th)
\end{verbatim}
\end{Usage}
%
\begin{Arguments}
\begin{ldescription}
\item[\code{theta}] ETAS parameters data.frame(mu=mu, K=K, alpha=alpha, c=c, p=p)

\item[\code{th}] 
\end{ldescription}
\end{Arguments}
\inputencoding{utf8}
\HeaderA{Inv.IntInjecIntensity}{Title}{Inv.IntInjecIntensity}
%
\begin{Description}\relax
Title
\end{Description}
%
\begin{Usage}
\begin{verbatim}
Inv.IntInjecIntensity(a = 50, V.i = 1, tau = 10, T.i, number.injected.events)
\end{verbatim}
\end{Usage}
%
\begin{Arguments}
\begin{ldescription}
\item[\code{a}] Event rate per unit volume injected

\item[\code{V.i}] Injected volume

\item[\code{tau}] Decau rate \LinkA{days}{days}

\item[\code{T.i}] Time of injection event

\item[\code{number.injected.events}] 
\end{ldescription}
\end{Arguments}
\inputencoding{utf8}
\HeaderA{inv.loggaus.t}{Inverse log-gaussian copula transformation:}{inv.loggaus.t}
%
\begin{Description}\relax
Inverse log-gaussian copula transformation:
\end{Description}
%
\begin{Usage}
\begin{verbatim}
inv.loggaus.t(x, m, s)
\end{verbatim}
\end{Usage}
%
\begin{Arguments}
\begin{ldescription}
\item[\code{s}] 
\end{ldescription}
\end{Arguments}
\inputencoding{utf8}
\HeaderA{inv.unif.t}{Inverse uniform copula transformation:}{inv.unif.t}
%
\begin{Description}\relax
Inverse uniform copula transformation:
\end{Description}
%
\begin{Usage}
\begin{verbatim}
inv.unif.t(x, a, b)
\end{verbatim}
\end{Usage}
%
\begin{Arguments}
\begin{ldescription}
\item[\code{b}] 
\end{ldescription}
\end{Arguments}
\inputencoding{utf8}
\HeaderA{It\_df}{Title}{It.Rul.df}
%
\begin{Description}\relax
Title
\end{Description}
%
\begin{Usage}
\begin{verbatim}
It_df(param_, time.df)
\end{verbatim}
\end{Usage}
%
\begin{Arguments}
\begin{ldescription}
\item[\code{time.df}] 
\end{ldescription}
\end{Arguments}
\inputencoding{utf8}
\HeaderA{lambda.N}{Title}{lambda.N}
%
\begin{Description}\relax
Title
\end{Description}
%
\begin{Usage}
\begin{verbatim}
lambda.N(th.mu, th.K, th.alpha, th.c, th.p, T1, T2, M0, Ht, link.functions)
\end{verbatim}
\end{Usage}
%
\begin{Arguments}
\begin{ldescription}
\item[\code{link.functions}] 
\end{ldescription}
\end{Arguments}
\inputencoding{utf8}
\HeaderA{lambda\_}{Conditional intensity - used by bayesianETAS Used for comparing output of inlabru with Bayesian ETAS}{lambda.Rul.}
%
\begin{Description}\relax
Conditional intensity - used by bayesianETAS
Used for comparing output of inlabru with Bayesian ETAS
\end{Description}
%
\begin{Usage}
\begin{verbatim}
lambda_(th, t, ti.v, mi.v, M0)
\end{verbatim}
\end{Usage}
%
\begin{Arguments}
\begin{ldescription}
\item[\code{M0}] Minimum magnitude threshold
\end{ldescription}
\end{Arguments}
\inputencoding{utf8}
\HeaderA{lambda\_2}{conditional intensity (used by Inlabru)}{lambda.Rul.2}
%
\begin{Description}\relax
conditional intensity (used by Inlabru)
\end{Description}
%
\begin{Usage}
\begin{verbatim}
lambda_2(th, t, ti.v, mi.v, M0)
\end{verbatim}
\end{Usage}
%
\begin{Arguments}
\begin{ldescription}
\item[\code{th}] Set of trial ETAS parameters ??

\item[\code{M0}] Minimum magnitude threshold
\end{ldescription}
\end{Arguments}
\inputencoding{utf8}
\HeaderA{log.Lambda\_h}{integrated triggering function - used by bayesianETAS Used for comparing output of inlabru with Bayesian ETAS}{log.Lambda.Rul.h}
%
\begin{Description}\relax
integrated triggering function - used by bayesianETAS
Used for comparing output of inlabru with Bayesian ETAS
\end{Description}
%
\begin{Usage}
\begin{verbatim}
## S3 method for class 'Lambda_h'
log(th, ti, mi, M0, T1, T2)
\end{verbatim}
\end{Usage}
%
\begin{Arguments}
\begin{ldescription}
\item[\code{M0}] Minimum magnitude threshold

\item[\code{T1}] Start of temporal model domain.

\item[\code{T2}] End of temporal model domain.
\end{ldescription}
\end{Arguments}
\inputencoding{utf8}
\HeaderA{log.Lambda\_h2}{integrated triggering function - used by Inlabru ALSO USED IN GENERATION SAMPLES}{log.Lambda.Rul.h2}
%
\begin{Description}\relax
integrated triggering function - used by Inlabru ALSO USED IN GENERATION SAMPLES
\end{Description}
%
\begin{Usage}
\begin{verbatim}
## S3 method for class 'Lambda_h2'
log(theta, ti, mi, M0, T1, T2)
\end{verbatim}
\end{Usage}
%
\begin{Arguments}
\begin{ldescription}
\item[\code{theta}] ETAS parameters \code{data.frame(mu=mu, K=K, alpha=alpha, c=c, p=p)}.

\item[\code{ti}] Time of parent event.

\item[\code{mi}] Magnitude of parent event

\item[\code{M0}] Minimum magnitude threshold

\item[\code{T1}] Start of temporal model domain.

\item[\code{T2}] End of temporal model domain.
\end{ldescription}
\end{Arguments}
\inputencoding{utf8}
\HeaderA{loggaus.t}{Log-gaussian copula transformation: Conversion of ETAS para to internal scale}{loggaus.t}
%
\begin{Description}\relax
Log-gaussian copula transformation: Conversion of ETAS para to internal scale
\end{Description}
%
\begin{Usage}
\begin{verbatim}
loggaus.t(x, m, s)
\end{verbatim}
\end{Usage}
%
\begin{Arguments}
\begin{ldescription}
\item[\code{s}] 
\end{ldescription}
\end{Arguments}
\inputencoding{utf8}
\HeaderA{omori\_plot}{Code to plot samples from the of posterior ETAS triggering function}{omori.Rul.plot}
%
\begin{Description}\relax
Code to plot samples from the of posterior ETAS triggering function
\end{Description}
%
\begin{Usage}
\begin{verbatim}
omori_plot(list.input, n.samp = 10, t.end = 1, n.breaks = 100)
\end{verbatim}
\end{Usage}
%
\begin{Arguments}
\begin{ldescription}
\item[\code{list.input}] 

\item[\code{n.breaks}] 
\end{ldescription}
\end{Arguments}
\inputencoding{utf8}
\HeaderA{Plot\_grid}{Title}{Plot.Rul.grid}
%
\begin{Description}\relax
Title
\end{Description}
%
\begin{Usage}
\begin{verbatim}
Plot_grid(
  xx = xx.,
  yy = yy.,
  delta_ = delta.,
  n.layer = n.layer.,
  bdy_ = bdy.,
  min.edge = min.edge.
)
\end{verbatim}
\end{Usage}
%
\begin{Arguments}
\begin{ldescription}
\item[\code{min.edge}] 
\end{ldescription}
\end{Arguments}
\inputencoding{utf8}
\HeaderA{post\_sampling}{Function to return a many (n.samp) samples from the posterior of the parameters}{post.Rul.sampling}
%
\begin{Description}\relax
Function to return a many (n.samp) samples from the posterior of the parameters
\end{Description}
%
\begin{Usage}
\begin{verbatim}
post_sampling(input.list, n.samp)
\end{verbatim}
\end{Usage}
%
\begin{Arguments}
\begin{ldescription}
\item[\code{input.list}] Which has combined the input file (for link functions) and bru output (for marginals)

\item[\code{n.samp}] The number of samples to draw from the posteriors
\end{ldescription}
\end{Arguments}
%
\begin{Value}
n.samp samples drawn from the posteriors.
\end{Value}
\inputencoding{utf8}
\HeaderA{sample.GR.magnitudes}{Return a sample of magnitudes drawn from the GR distribution}{sample.GR.magnitudes}
%
\begin{Description}\relax
Return a sample of magnitudes drawn from the GR distribution
\end{Description}
%
\begin{Usage}
\begin{verbatim}
sample.GR.magnitudes(n, beta.p, M0)
\end{verbatim}
\end{Usage}
%
\begin{Arguments}
\begin{ldescription}
\item[\code{n}] Number of events in the sample.

\item[\code{beta.p}] Related to the b-value via \AsIs{\texttt{b ln(10)}}.

\item[\code{M0}] Minimum magnitude for the sample.
\end{ldescription}
\end{Arguments}
%
\begin{Value}
A list of magnitudes of length \code{n} drawn from a GR distribution.
\end{Value}
%
\begin{Examples}
\begin{ExampleCode}
sample.GR.magnitudes(n=100, beta.p=log(10), M0=2.5)
\end{ExampleCode}
\end{Examples}
\inputencoding{utf8}
\HeaderA{sample.temoral.ETAS.daughters}{Generate a sample of new events \code{data.frame(t\_i, M\_i)} of length \code{n.ev} for one parent event occuring at time \code{t\_h} using the ETAS model.}{sample.temoral.ETAS.daughters}
%
\begin{Description}\relax
Generate a sample of new events \code{data.frame(t\_i, M\_i)} of length \code{n.ev} for one parent event occuring at time \code{t\_h} using the ETAS model.
\end{Description}
%
\begin{Usage}
\begin{verbatim}
sample.temoral.ETAS.daughters(theta, beta.p, th, n.ev, M0, T1, T2)
\end{verbatim}
\end{Usage}
%
\begin{Arguments}
\begin{ldescription}
\item[\code{theta}] ETAS parameters \code{data.frame(mu=mu, K=K, alpha=alpha, c=c, p=p)}.

\item[\code{beta.p}] Slope of GR relation: beta = b ln(10).

\item[\code{th}] Time of parent event \LinkA{days}{days}.

\item[\code{n.ev}] The number of events to be placed.

\item[\code{M0}] Minimum magnitude in synthetic catalogue.

\item[\code{T1}] Start time for synthetic catalogue \LinkA{days}{days}.

\item[\code{T2}] End time for synthetic catalogue \LinkA{days}{days}.
\end{ldescription}
\end{Arguments}
%
\begin{Value}
Generate a sample of new events \code{data.frame(t\_i, M\_i)} from one parent
\end{Value}
\inputencoding{utf8}
\HeaderA{sample.temoral.injection.events}{Title}{sample.temoral.injection.events}
%
\begin{Description}\relax
Title
\end{Description}
%
\begin{Usage}
\begin{verbatim}
sample.temoral.injection.events(a = 50, V.i = 1, tau = 10, beta.p, M0, T.i, T2)
\end{verbatim}
\end{Usage}
%
\begin{Arguments}
\begin{ldescription}
\item[\code{a}] Induced event rate per unit volume.

\item[\code{V.i}] Injected volume

\item[\code{tau}] Decay rate \LinkA{days}{days}.

\item[\code{beta.p}] Related to the b-value via \AsIs{\texttt{b ln(10)}}.

\item[\code{M0}] Minimum magnitude threshold.

\item[\code{T.i}] Time of injection \LinkA{days}{days}.

\item[\code{T2}] End of temporal model domain \LinkA{days}{days}.
\end{ldescription}
\end{Arguments}
%
\begin{Value}
Catalogue of parent events induced by injection data.frame(times, magnitudes)
\end{Value}
\inputencoding{utf8}
\HeaderA{sample.temporal.ETAS.generation}{Take all previous parent events from \code{Ht=data.frame[ts, magnitudes]} and generates their daughters events using the ETAS model}{sample.temporal.ETAS.generation}
%
\begin{Description}\relax
Take all previous parent events from \code{Ht=data.frame[ts, magnitudes]} and generates their daughters events using the ETAS model
\end{Description}
%
\begin{Usage}
\begin{verbatim}
sample.temporal.ETAS.generation(theta, beta.p, Ht, M0, T1, T2, ncore = 1)
\end{verbatim}
\end{Usage}
%
\begin{Arguments}
\begin{ldescription}
\item[\code{theta}] ETAS parameters \code{data.frame(mu=mu, K=K, alpha=alpha, c=c, p=p)}.

\item[\code{beta.p}] Slope of GR relation: beta = b ln(10).

\item[\code{Ht}] The set of parent events in the form \code{data.frame[ts, magnitudes]}

\item[\code{M0}] The minimum earthquake magnitude in the synthetic catalogue.

\item[\code{T1}] The start time for the synthetic catalogue \LinkA{days}{days}.

\item[\code{T2}] The end time for the synthetic catalogue \LinkA{days}{days}.

\item[\code{ncore}] The number of compute cores to use
\end{ldescription}
\end{Arguments}
%
\begin{Value}
Return one generation of daughters from the parents in \code{Ht} in the form \code{data.frame(t\_i, M\_i)}.
\end{Value}
%
\begin{Examples}
\begin{ExampleCode}
# The parents are specified in Ht
Ht <- data.frame(ts=c(500), magnitudes=c(6.7))
sample.temporal.ETAS.generation( theta=data.frame(mu=0.1, K=0.089, alpha=2.29, c=0.11, p=1.08), beta.p=log(10), M0=2.5, T1=0, T2=1000, Ht=Ht )
\end{ExampleCode}
\end{Examples}
\inputencoding{utf8}
\HeaderA{sample.temporal.ETAS.times}{Sampling times for events triggered by a parent at th according to the ETAS triggering function}{sample.temporal.ETAS.times}
%
\begin{Description}\relax
Sampling times for events triggered by a parent at th according to the ETAS triggering function
\end{Description}
%
\begin{Usage}
\begin{verbatim}
sample.temporal.ETAS.times(theta, n.ev, th, T2)
\end{verbatim}
\end{Usage}
%
\begin{Arguments}
\begin{ldescription}
\item[\code{theta}] ETAS parameters \code{data.frame(mu=mu, K=K, alpha=alpha, c=c, p=p)}.

\item[\code{n.ev}] Number of events to return in the sample in time domain (th, T2].

\item[\code{th}] Time of the parent event producing n.ev daughters.

\item[\code{T2}] End time of model domain.
\end{ldescription}
\end{Arguments}
%
\begin{Value}
t.sample A list of times in the interval \LinkA{0, T2}{0, T2} distributed according to the ETAS triggering function.
\end{Value}
\inputencoding{utf8}
\HeaderA{temporal.ETAS}{Function to fit Hawkes process model}{temporal.ETAS}
%
\begin{Description}\relax
function to fit a temporal ETAS model using \code{inlabru}.
\end{Description}
%
\begin{Usage}
\begin{verbatim}
temporal.ETAS(
  sample.s,
  M0,
  T1,
  T2,
  link.functions = NULL,
  coef.t.,
  delta.t.,
  N.max.,
  bru.opt
)
\end{verbatim}
\end{Usage}
%
\begin{Arguments}
\begin{ldescription}
\item[\code{sample.s}] Observed events: \code{data.frame} with columns time (ts), magnitude (magnitudes), event identifier (idx.p). Column names must not be changed.

\item[\code{M0}] Minimum magnitude threshold, \code{scalar}

\item[\code{T1}] Start of temporal model domain, \code{scalar} \LinkA{measure unit of sample.s\$ts}{measure unit of sample.s.Rdol.ts}.

\item[\code{T2}] End of temporal model domain, \code{scalar} \LinkA{measure unit of sample.s\$ts}{measure unit of sample.s.Rdol.ts}.

\item[\code{link.functions}] Functions to transform the parameters from the internal INLA scale to the ETAS scale. It must be a \code{list} of functions with names (mu, K, alpha, c\_, p)

\item[\code{coef.t.}] TimeBinning parameter: parameter regulating the relative length of successive bins, \code{scalar}.

\item[\code{delta.t.}] TimeBinning parameter: parameter regulating the bins' width, \code{scalar}.

\item[\code{N.max.}] TimeBinning parameter: parameter regulating the Number of bins (= \code{N.max} + 2), \code{scalar}.

\item[\code{bru.opt}] Runtime options for inlabru: See https://inlabru-org.github.io/inlabru/reference/bru\_call\_options.html, \code{list}
\end{ldescription}
\end{Arguments}
%
\begin{Value}
The fitted model as a 'bru' object, which is a list
\end{Value}
\inputencoding{utf8}
\HeaderA{Temporal.ETAS.fit}{Fits the remporal ETAS model and returns the results. This function decomposes the input.list for the `Hawkes.bru2`` function.}{Temporal.ETAS.fit}
%
\begin{Description}\relax
Fits the remporal ETAS model and returns the results. This function decomposes the input.list for the `Hawkes.bru2`` function.
\end{Description}
%
\begin{Usage}
\begin{verbatim}
Temporal.ETAS.fit(input.list)
\end{verbatim}
\end{Usage}
%
\begin{Arguments}
\begin{ldescription}
\item[\code{input.list}] All input data and parameters are passed to inlabru via this structured list.
\end{ldescription}
\end{Arguments}
%
\begin{Value}
The fitted model as a bru object, which is a list
\end{Value}
\inputencoding{utf8}
\HeaderA{time.grid}{Generate a set of time bins for a specific event and return.}{time.grid}
%
\begin{Description}\relax
Generate a set of time bins for a specific event and return.
\end{Description}
%
\begin{Usage}
\begin{verbatim}
## S3 method for class 'grid'
time(data.point, coef.t, delta.t, T2., displaygrid = FALSE, N.exp.)
\end{verbatim}
\end{Usage}
%
\begin{Arguments}
\begin{ldescription}
\item[\code{coef.t}] TimeBinning parameter:

\item[\code{delta.t}] TimeBinning parameter:

\item[\code{T2.}] End of the temporal domain \LinkA{days}{days}.

\item[\code{displaygrid}] Boolean variable - whether to plot the grid

\item[\code{N.exp.}] TimeBinning parameter:
\end{ldescription}
\end{Arguments}
%
\begin{Value}
A set of time bins aggregated over all events MORE DETAIL
\end{Value}
%
\begin{Examples}
\begin{ExampleCode}
## EXAMPLE 1
events <- data.frame( ts=c(0,1 , 3 ), idx.p=c(1,2,3) )
T2 <- 20
N.exp <- 8
delta.t <- 0.1
coef.t <- 1
time.grid(events, coef.t, delta.t, T2, displaygrid = FALSE, N.exp)
\end{ExampleCode}
\end{Examples}
\inputencoding{utf8}
\HeaderA{unif.t}{Uniform copula transformation: Conversion of ETAS para to internal scale}{unif.t}
%
\begin{Description}\relax
Uniform copula transformation: Conversion of ETAS para to internal scale
\end{Description}
%
\begin{Usage}
\begin{verbatim}
unif.t(x, a, b)
\end{verbatim}
\end{Usage}
%
\begin{Arguments}
\begin{ldescription}
\item[\code{b}] 
\end{ldescription}
\end{Arguments}
\printindex{}
\end{document}
